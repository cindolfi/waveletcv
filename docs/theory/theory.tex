\documentclass[12pt]{article}
\usepackage{amsmath}
\usepackage{amssymb}

\newcommand{\Fourier}[1]{\mathcal{F}[#1]}
\newcommand{\Wavelet}[1]{\mathcal{W}[#1]}
\newcommand{\DWT}[1]{\mathrm{W}[#1]}
\newcommand{\innerprod}[2]{\langle #1, #2 \rangle}
\newcommand{\Zset}{\mathbb{Z}}

% \newcommand{\R}{\mathbb{R}}
% \newcommand{\C}{\mathbb{C}}
 
% https://rafat.github.io/sites/wavebook/intro/mra.html

\begin{document}

\section{Introduction}

The Fourier transform yields a frequency domain representation that provides a picture of the \textit{global shape} of a signal, but suffers from a total lack of \textit{locality in time}.
Conversely, when viewed in the time domain we get perfect knowledge of the magnitude of the signal at each \textit{instant of time}, but no information about its \textit{shape}.
The wavelet transform casts a signal to domain that can understood as a tradeoff between time and scale (i.e inverse frequency).  
That is, it provides information about the \textit{local shape} of the signal over an \textit{interval of time}.

To be more concrete, the Fourier transform of $f(t)$ is projection onto the set of basis functions formed by varying the complex frequency $\omega$ of exponentials with unit magnitude.
\begin{equation}
    \Fourier{f}(\omega) = \innerprod{f}{e^{-i\omega t}} = \int_{-\infty}^{\infty} f(\tau) e^{-i\omega \tau} d\tau
\end{equation}
Likewise, albeit somewhat vacuousaly, we can understand a function's time domain representation of the as the inner product between a function and the set of standard basis functions $\delta_\tau$.
\begin{equation}
    f(t) = \innerprod{f}{\delta_\tau} \int_{-\infty}^{\infty} f(\tau) \delta(t - \tau) d\tau
\end{equation}
The continuous wavelet transform is the projection of $f(t)$ onto the basis generated by translations and dilations of a given $\psi$ called the mother wavelet,
\begin{equation}
    \Wavelet{f}(k, s) = \innerprod{f}{\psi_{k, s}} = \int_{-\infty}^{\infty} f(\tau) \psi_{k, s}(t) d\tau
\end{equation}
where
\begin{equation}
    \psi_{k, s}(t) = \frac{1}{\sqrt{s}} \psi(\frac{t - k}{s})
\end{equation}

\subsection{Common Wavelet Families}

\subsubsection{Haar}
\subsubsection{Daubechies}

\subsection{Wavelet Properties}



\section{Discrete Wavelet Transform}

A naive discretization of the continuous wavelet transform where the translation and scale parameters are taken over small, equally space steps is not only computationally expensive, it is also overcomplete in the sense that it contains more information than necessary for the transform to be invertible.

A better approach is to let $s = a^m$ and $k = a^m n$, where $m, n \in \Zset$ are the discrete scale and time variables, respectively, and $a > 1$ is an arbitarily chosen parameter.  We define the discrete family of basis functions as
\begin{equation}
    \psi_{m, n}(t) = a^{\frac{-m}{2}} \psi(a^{-m}t - n)
\end{equation}
When we take $a = 2$ we get the standard discrete wavelet transform ($DWT$) basis
\begin{equation}
    \psi_{m, n}(t) = 2^{\frac{-m}{2}} \psi(2^{-m}t - n)
\end{equation}
where the support of each basis function yields a dyadic tiling of the time-scale domain.

When viewed in the frequency domain, the inner product between a function and any wavelet basis function is a band pass filter.
Since the discrete scaling parameter $j_0 \le j \le J$, the discretization given above must necessarily discard signal information for frequencies below the longest wavelet scale $j_0$.  In other words the discretization fails to represent signal frequencies smaller that those in the Fourier transform of $\psi_{j_0, n}(t)$.
We rectify this by introducing a second function $\phi$, called the scaling function, that performs the requiste low pass filtering to recover this portion of the signal.
\begin{equation}
    \DWT{f} = \innerprod{f}{\psi_{j_0, n}} + \sum_{j = j_0}^{J} \innerprod{f}{\psi_{j, n}}
\end{equation}
With this we can define the inverse discrete wavelet transform (IDWT) by letting $\hat{\phi}_{j_0, n}$ and $\hat{\psi}_{j, n}$ be functions such that the function $f(t)$ can be synthesized from its wavelet coefficients
\begin{equation}
    f(t) = \sum_n \innerprod{f}{\psi_{j_0, n}} \hat{\phi}_{j_0, n}(t) + \sum_{j = j_0}^{J} \innerprod{f}{\psi_{j, n}} \hat{\psi}_{j, n}(t)
\end{equation}

\section{Wavelet Filters}

\begin{enumerate}
    \item Vanising Moments
    \item Regularity
    \item Admissibility Condition
    \item Compact Support
    \item Symmetry
    \item Orthogonality
\end{enumerate}


\end{document}

